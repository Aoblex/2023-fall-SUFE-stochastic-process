\section{随机过程的基本概念}
    \subsection{随机过程的定义与有穷维分布族} 
        \begin{definition}[随机过程]

            给定概率空间$(\Omega, \mathscr{F}, P)$及指标集$\mathbb{T} \neq  \emptyset $,若$\forall t, \forall c \in \mathbb{R}, \{\omega | \textbf{X}_t(\omega) \leq c\} \in \mathscr{F}$,则称$\{X_t(\omega), t \in \mathbb{T}\}$ 为随机过程$(Stochastic\ Process)$。

        \end{definition}

        \begin{definition}[样本轨道]
            
            随机过程 \StochasticProcess 是关于$t \in \mathbb{T}$和$\omega \in \Omega$的二元函数,当$\omega$固定,$X(\cdot, \omega)$是$t \in \mathbb{T}$的函数,称为样本轨道$(Sample\ Path)$。

        \end{definition}

        \begin{definition}[有穷维分布族]

            给定实值随机过程 \StochasticProcess ,对于$\forall n \geq 1, \forall {\{t_i\}}_{i=1}^{n} \subset \mathbb{T} $,可得$(\textbf{X}_{t_1}, \cdots, \textbf{X}_{t_n})$的联合分布函数为:
            $$
            F_{t_1, \cdots, t_n}(x_{t_1}, \cdots, x_{t_n}) = P\{\textbf{X}_{t_1} \leq x_{t_1}, \cdots, \textbf{X}_{t_n} \leq x_{t_n}\}
            $$

            \noindent
            有穷维分布函数族$\mathscr{D} \triangleq \{F_{t_1, \cdots, t_n}(x_{t_1}, \cdots, x_{t_n})| \forall n \geq 1, \forall \{t_i\}_{i=1}^{n} \subset \mathbb{T}\}$
            
            \noindent
            定义$(\textbf{X}_{t_1}, \cdots, \textbf{X}_{t_n})$的联合矩母函数为:
            $$
            \varphi_{t_1, \cdots, t_n}(u_{t_1}, \cdots, u_{t_n}) = E\left[e^{\sum_{j=1}^{n} u_{t_j} \textbf{X}_{t_j}}\right]
            $$
            有穷维矩母函数族$\mathscr{C} \triangleq \{\varphi_{t_1, \cdots, t_n}(u_{t_1}, \cdots, u_{t_n}) | \forall n \geq 1, \forall \{t_i\}_{i=1}^{n} \subset \mathbb{T}\}$

        \end{definition}

        \begin{definition}[独立随机过程]
            随机过程 \StochasticProcess 满足:
            $$
            F_{t_1, \cdots, t_n}(x_{t_1}, \cdots, x_{t_n}) = \prod_{k=1}^{n} F_{t_k}(x_{t_k})\ \ \ \ (\forall n \geq 1, \forall \{t_i\}_{i=1}^{n})
            $$
            则称$\{\textbf{X}_t, t \in \mathbb{T}\}$为独立随机过程。
        \end{definition}

        \begin{definition}[均值函数]
            给定随机过程 \StochasticProcess,定义均值函数为:
            $$
            m(t) = E(\textbf{X}_t)
            $$
        \end{definition}

        \begin{definition}[方差函数]
            给定随机过程 \StochasticProcess, 定义方差函数为:
            $$
            D(t) = Var(\textbf{X}_t) = E{\left(\textbf{X}_t - m(t)\right)}^2
            $$
        \end{definition}

        \begin{definition}[自相关函数]
            给定随机过程 \StochasticProcess,定义自相关函数为:
            $$
            R(s, t) = E(\textbf{X}_s\textbf{X}_t)
            $$
        \end{definition}

        \begin{definition}[协方差函数]
            给定随机过程 \StochasticProcess, 定义协方差函数为:
            $$
            k(s, t) = cov(\textbf{X}_s \textbf{X}_t) = E((\textbf{X}_s - m(s))(\textbf{X}_t - m(t))) = R(s, t) - m(s)m(t)
            $$
        \end{definition}
    \subsection{随机过程的分类}
        \begin{definition}[独立增量过程]
            若 \StochasticProcess 满足:对于 $\forall 0 \leq t_1 < t_2 < \cdots < t_{n - 1} < t_{n}$ ,有 $\textbf{X}_{t_2} - \textbf{X}_{t_1}, \textbf{X}_{t_3} - \textbf{X}_{t_2}, \cdots, \textbf{X}_{t_n} - \textbf{X}_{t_{n - 1}}$ 相互独立,则称 \StochasticProcess 为独立增量过程。
        \end{definition}

        \begin{definition}[平稳增量过程]
            若 \StochasticProcess 满足:对于 $ \forall t, h \geq 0\ (t, t + h \in \mathbb{T}) $ ,有 $ \textbf{X}_{t + h} - \textbf{X}_{t}$ 的分布与 $t$ 无关,则称 \StochasticProcess 为平稳增量过程。
        \end{definition}

        \begin{definition}[平稳独立增量过程]
            若 \StochasticProcess 既是独立增量过程,又是平稳增量过程,则称 \StochasticProcess 是平稳独立增量过程。
        \end{definition}

        \begin{definition}[计数过程]
            \CountingProcess 称为一个计数过程,若:
            \begin{enumerate}[label=(\arabic*).]
                \item $\textbf{N}(0) = 0$
                \item $\textbf{N}(t) \in \mathbb{N}, \forall t \geq 0$
                \item $\textbf{N}(s) \leq \textbf{N}(t), \forall s < t$
                \item 当 $s < t$ , $\textbf{N}(t) - \textbf{N}(s)$ 等于 $(s, t]$ 中发生的事件的个数。
            \end{enumerate}
        \end{definition}

        \begin{definition}[正态过程]
           如果 \StochasticProcess 对于 $ \forall n \geq 1, \forall \{ t_i \}_{i = 1}^{n} \subset \mathbb{T}$ 有 $\left( \textbf{X}_{t_1}, \cdots, \textbf{X}_{t_n}\right) \sim N(\mu, \Sigma)$,则称 \StochasticProcess 为正态过程。
        \end{definition}

        \begin{definition}[弱平稳过程/宽平稳过程/协方差平稳过程]
            如果 \StochasticProcess 满足:
            \begin{enumerate}[label=(\arabic*)]
                \item $m(t) \equiv C\ (C \in \mathbb{R})$
                \item $k(s, s+t) = B(t)$
                \item $ E\left( \textbf{X}_t^2 \right) < +\infty $
            \end{enumerate}
            则称 \StochasticProcess 为弱平稳过程/宽平稳过程/协方差平稳过程。
        \end{definition}

        \begin{definition}[强平稳过程/严平稳过程/狭义平稳过程]
            如果 \StochasticProcess 对 $\forall n \in \mathbb{N}^*, \forall t_1 < t_2 < \cdots < t_n, \forall h$,有:
            $$
            P\left( \textbf{X}_{t_1} \leq \lambda_1, \cdots, \textbf{X}_{t_n} \leq \lambda_n \right) = P\left( \textbf{X}_{t_1 + h} \leq \lambda_1, \cdots, \textbf{X}_{t_n + h} \leq \lambda_n \right)
            $$
            即$(\textbf{X}_{t_1}, \cdots, \textbf{X}_{t_n})$与$(\textbf{X}_{t_1 + h}, \cdots, \textbf{X}_{t_n + h})$同分布,则称 \StochasticProcess 为强平稳过程/严平稳过程/狭义平稳过程。
            
        \end{definition}