\section{随机过程的基本概念}
    \subsection{随机过程的定义与有穷维分布族} 
        \begin{definition}[随机过程]

            给定概率空间$(\Omega, \mathscr{F}, P)$及指标集$\mathbb{T} \neq  \emptyset $,若$\forall t, \forall c \in \mathbb{R}, \{\omega | \textbf{X}_t(\omega) \leq c\} \in \mathscr{F}$,则称$\{X_t(\omega), t \in \mathbb{T}\}$ 为随机过程$(Stochastic\ Process)$。

        \end{definition}

        \begin{definition}[样本轨道]
            
            随机过程 \StochasticProcess 是关于$t \in \mathbb{T}$和$\omega \in \Omega$的二元函数,当$\omega$固定,$X(\cdot, \omega)$是$t \in \mathbb{T}$的函数,称为样本轨道$(Sample\ Path)$。

        \end{definition}

        \begin{definition}[有穷维分布族]

            给定实值随机过程 \StochasticProcess ,对于$\forall n \geq 1, \forall {\{t_i\}}_{i=1}^{n} \subset \mathbb{T} $,可得$(\textbf{X}_{t_1}, \cdots, \textbf{X}_{t_n})$的联合分布函数为:
            $$
            F_{t_1, \cdots, t_n}(x_{t_1}, \cdots, x_{t_n}) = P\{\textbf{X}_{t_1} \leq x_{t_1}, \cdots, \textbf{X}_{t_n} \leq x_{t_n}\}
            $$

            \noindent
            有穷维分布函数族$\mathscr{D} \triangleq \{F_{t_1, \cdots, t_n}(x_{t_1}, \cdots, x_{t_n})| \forall n \geq 1, \forall \{t_i\}_{i=1}^{n} \subset \mathbb{T}\}$
            
            \noindent
            定义$(\textbf{X}_{t_1}, \cdots, \textbf{X}_{t_n})$的联合矩母函数为:
            $$
            \varphi_{t_1, \cdots, t_n}(u_{t_1}, \cdots, u_{t_n}) = E\left[e^{\sum_{j=1}^{n} u_{t_j} \textbf{X}_{t_j}}\right]
            $$
            有穷维矩母函数族$\mathscr{C} \triangleq \{\varphi_{t_1, \cdots, t_n}(u_{t_1}, \cdots, u_{t_n}) | \forall n \geq 1, \forall \{t_i\}_{i=1}^{n} \subset \mathbb{T}\}$

        \end{definition}

        \begin{definition}[独立随机过程]
            随机过程 \StochasticProcess 满足:
            $$
            F_{t_1, \cdots, t_n}(x_{t_1}, \cdots, x_{t_n}) = \prod_{k=1}^{n} F_{t_k}(x_{t_k})\ \ \ \ (\forall n \geq 1, \forall \{t_i\}_{i=1}^{n})
            $$
            则称$\{\textbf{X}_t, t \in \mathbb{T}\}$为独立随机过程。
        \end{definition}

        \begin{definition}[均值函数]
            给定随机过程 \StochasticProcess,定义均值函数为:
            $$
            m(t) = E(\textbf{X}_t)
            $$
        \end{definition}

        \begin{definition}[方差函数]
            给定随机过程 \StochasticProcess, 定义方差函数为:
            $$
            D(t) = Var(\textbf{X}_t) = E{\left(\textbf{X}_t - m(t)\right)}^2
            $$
        \end{definition}

        \begin{definition}[自相关函数]
            给定随机过程 \StochasticProcess,定义自相关函数为:
            $$
            R(s, t) = E(\textbf{X}_s\textbf{X}_t)
            $$
        \end{definition}

        \begin{definition}[协方差函数]
            给定随机过程 \StochasticProcess, 定义协方差函数为:
            $$
            k(s, t) = cov(\textbf{X}_s \textbf{X}_t) = E((\textbf{X}_s - m(s))(\textbf{X}_t - m(t))) = R(s, t) - m(s)m(t)
            $$
        \end{definition}
    \subsection{随机过程的分类}
        \begin{definition}
            
        \end{definition}