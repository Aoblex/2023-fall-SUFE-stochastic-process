\section{随机过程的基本概念}
    \subsection{随机过程的定义与有穷维分布族} 
        \begin{definition}[随机过程]

            给定概率空间$(\Omega, \mathscr{F}, P)$及指标集$\mathbb{T} \neq  \emptyset $,若$\forall t, \forall c \in \mathbb{R}, \{\omega | \textbf{X}_t(\omega) \leq c\} \in \mathscr{F}$,则称$\{X_t(\omega), t \in \mathbb{T}\}$ 为随机过程$(Stochastic\ Process)$。

        \end{definition}

        \begin{definition}[样本轨道]
            
            随机过程$X(t, \omega)$是关于$t \in \mathbb{T}$和$\omega \in \Omega$的二元函数,当$\omega$固定,$X(\cdot, \omega)$是$t \in \mathbb{T}$的函数,称为样本轨道$(Sample\ Path)$。

        \end{definition}

        \begin{definition}[有穷维分布族]
            
        \end{definition}
    \subsection{随机过程的分类}
        \begin{definition}
            
        \end{definition}