\section{泊松过程}
    \subsection{泊松过程的定义}
        \begin{definition}[泊松过程]
            计数过程 \CountingProcess 称为参数为$\lambda(\lambda > 0)$ 的泊松过程,若满足:
            \begin{enumerate}[label=(\arabic*).]
                \item \CountingProcess 是独立增量过程
                \item $\forall s, t \geq 0$,有 $$ P\left\{ \textbf{N}(s + t) - \textbf{N}(s) = n \right\} = e^{-\lambda t} \frac{(\lambda t)^n}{n !} , n = 0, 1, 2, \cdots $$
            \end{enumerate}
        \end{definition}

        \begin{definition}[泊松过程]
            计数过程 \CountingProcess 称为参数为 $\lambda (\lambda > 0)$ 的泊松过程,若满足:
            \begin{enumerate}[label=(\arabic*).]
                \item \CountingProcess 是独立增量过程
                \item \CountingProcess 是平稳增量过程
                \item 对于$\forall t>0$和足够小的$h > 0$,有:
                \begin{align*}
                    P\left\{ \textbf{N}(t + h) - \textbf{N}(t) = 1 \right\} &= \lambda h + o(h) \\
                    P\left\{ \textbf{N}(t + h) - \textbf{N}(t) \geq 2 \right\} &= o(h)
                \end{align*}

            \end{enumerate}
        \end{definition}
    \subsection{泊松过程的性质}
        \begin{definition}[分类泊松过程]
            假定在 $s$ 时刻发生的事件以概率 $P(s)$ 被归为1型,以概率 $ 1 - P(s) $ 被归为2型,且各个事件的归类相互独立。记 $\{\textbf{N}_i(t), t \geq 0\}, i = 1, 2$ 为 $t$ 时 $i$型事件发生的个数。
        \end{definition}
    \subsection{非齐次的泊松过程}
        \begin{definition}[非齐次泊松过程]
            计数过程 \CountingProcess 若满足:
            \begin{enumerate}[label=(\arabic*).]
                \item \CountingProcess 是独立增量过程
                \item $P\left\{ \textbf{N}(t + h) - \textbf{N}(t) = 1 \right\} = \lambda(t) h + o(h)$
                \item $P\left\{ \textbf{N}(t + h) - \textbf{N}(t) \geq 2 \right\} = o(h)$
            \end{enumerate}
            则称 \CountingProcess 是强度函数为$\lambda(t)\ (\lambda(t) > 0, t \geq 0)$ 的非齐次泊松过程。
        \end{definition}
    \subsection{复合泊松过程}
        \begin{definition}[复合泊松过程]
            $\{\textbf{Y}_k, k \in \mathbb{N}^* \}$ 独立同分布,\CountingProcess 是强度为 $\lambda$ 的泊松过程,且 \CountingProcess 与 $\{\textbf{Y}_k, k \in \mathbb{N}^* \}$ 独立,记:
            $$
            \textbf{X}(t) = \sum\limits_{k = 1}^{\textbf{N}(t)} \textbf{Y}_k\ (t \geq 0)
            $$
            则称 $\{\textbf{X}(t), t \geq 0\}$ 为复合泊松过程。
        \end{definition}