\section{离散时间的马尔可夫链}
    \subsection{马尔可夫链的基本概念}
        \begin{definition}[马尔可夫链]
            若 \MarkovChain 的状态空间和参数空间都是离散的,且具有马氏性:
            \begin{align*}
                &P\{ \textbf{X}_{n+1} = j | \textbf{X}_0 = i_0, \textbf{X}_1 = i_1, \cdots, \textbf{X}_{n - 1} = i_{n - 1}, \textbf{X}_n = i \} \\= &P\{ \textbf{X}_{n + 1} = j | \textbf{X}_n = i \}
            \end{align*}
            则称 \MarkovChain 为马氏链。
        \end{definition}

        \begin{definition}[一步转移概率]
            \MarkovChain 为马氏链,若 $P\{\textbf{X}_{n+1} = j | \textbf{X}_{n} = i\}$ 与 $n$ 无关,则:
            $$
            p_{ij} \triangleq P\{\textbf{X}_{n+1} = j | \textbf{X}_{n} = i\}
            $$
            称 \MarkovChain 为(时间)齐次的,称 $ p_{ij}$ 为 \MarkovChain 的一步转移概率,称 $P = \left[p_{ij}\right]$ 为 \MarkovChain 的一步转移概率矩阵。
        \end{definition}

        \begin{definition}[多步转移概率]
            称条件概率
            $$
            p_{ij}^{(m)} \triangleq P\{\textbf{X}_{n+m} = j | \textbf{X}_{n} = i\}
            $$ 
            为 \MarkovChain 的 $m$ 步转移概率,记 $P^{(m)} = \left[p_{ij}^{(m)}\right]$ 为 \MarkovChain 的 $m$ 步转移概率矩阵。
        \end{definition}

        \begin{definition}[概率分布行向量]
            对于 \MarkovChain,记 $n$ 时刻的分布行向量为:
            $$
            s(n) = \{s_i(n), i \in \mathbb{E} \}
            $$
            其中:
            $$
            s_i(n) = P(\textbf{X}_n = i)
            $$
        \end{definition}
    \subsection{马氏链的状态分类}
        \begin{definition}[可达]
            \MarkovChain 是一马氏链,给定任意的 $i, j \in \mathbb{E}$,若:
            $$
            \{n | p_{ij}^{(n)} > 0\} \neq \emptyset
            $$
            则称 $i$ 可达 $j$,记作 $ i \rightarrow j $。
        \end{definition}

        \begin{definition}[互通]
            \MarkovChain 是一马氏链,给定任意的 $i, j \in \mathbb{E}$,若:
            
            \begin{center}
                $i \rightarrow j$ 且 $j \rightarrow i$
            \end{center}

            \noindent
            则称 $i, j$ 互通, 记作 $ i \leftrightarrow j$
        \end{definition}
        
        \begin{definition}[状态分类]
            $\forall i, j \in \mathbb{E}$,若 $i \leftrightarrow j$,则称 $i, j$ 属于同一类状态。
        \end{definition}

        \begin{definition}[可约与不可约]
            若 \MarkovChain 只存在一个类,则称其是不可约的,否则称为可约的。
        \end{definition}

        \begin{definition}[可回态与不可回态]
            \MarkovChain 是一马氏链,给定任意的 $i \in \mathbb{E}$,若:
            $$
            P\{\textbf{X}_n = i | \textbf{X}_0 = i\} \equiv 0\ (\forall n \in \mathbb{N}^*)
            $$
            则称 $i$ 是不可回态,否则称为可回态。
        \end{definition}

        \begin{definition}[周期与非周期]
            \MarkovChain 是一马氏链, 给定任意的 $i \in \mathbb{E}$,记:
            $$
            d_i \triangleq \gcd \left(\{n | n \in \mathbb{N}^*, p_{ii}^{(n)} > 0\}\right)
            $$
            当 $d_i > 1$,称状态 $i$ 为周期的;当 $d_i = 1$, 称状态 $i$ 为非周期的。
        \end{definition}

        \begin{definition}[首达时间]
            \MarkovChain 是一马氏链,给定任意的 $i, j \in \mathbb{E}$,首达时间定义为:
            $$
            \textbf{T}_{ij} = \inf\{ n \geq 1 | \textbf{X}_n = j, \textbf{X}_0 = i \}
            $$
            若 $\{ n \geq 1 | \textbf{X}_n = j, \textbf{X}_0 = i \} = \emptyset$ ,则令 $ \textbf{T}_{ij} = \infty$。首达时间是一个随机变量。
        \end{definition}

        \begin{definition}[首达概率]
            \MarkovChain 是一马氏链,给定任意的$i, j \in \mathbb{E}$,首达概率定义为:
            \begin{align*}
                f_{ij}^{(n)} &= P\left\{ \textbf{T}_{ij} = n | \textbf{X}_0 = i \right\} \\
                &= P\left\{ \textbf{X}_n = j, \textbf{X}_k \neq j, 1 \leq k \leq n - 1 | \textbf{X}_0 = i \right\} 
            \end{align*}
            定义:
            \begin{align*}
                f_{ij}^{(0)} &= 0\ (i \neq j, i, j \in \mathbb{E}) \\
                f_{ii}^{(0)} &= 1\ ( i \in \mathbb{E})
            \end{align*} 
        \end{definition}

        \begin{definition}[可达概率]
            \MarkovChain 是一马氏链,给定任意的$i, j \in \mathbb{E}$,可达概率定义为:
            \begin{align*}
                f_{ij} &= P\{T_{ij} < +\infty | \textbf{X}_0 = i\} \\
                &= P\{\exists n, s.t.\ \textbf{X}_n = j, \textbf{X}_k \neq j, 1 \leq k \leq n - 1 | \textbf{X}_0 = i\} \\
                &= \sum\limits_{n = 1}^{\infty} f_{ij}^{(n)}
            \end{align*}
        \end{definition}

        \begin{definition}[平均可达时间]
            \MarkovChain 是一马氏链,给定任意的$i, j \in \mathbb{E}$,平均可达时间定义为:
            \[
                \mu_{ij} =
                \left\{
                    \begin{array}{l}
                        E\left[ T_{ij} \right] = \sum\limits_{n=1}^{\infty} n f_{ij}^{(n)}\ \ \ (f_{ij} = 1) \\
                        \infty\ \ \ (f_{ij} < 1)
                    \end{array}
                \right. 
            \] 
        \end{definition}

        \begin{definition}[常返]
            \MarkovChain 是一马氏链,给定任意的$i \in \mathbb{E}$,若 $f_{ii} = 1$,则称状态 $i$ 是常返的。
        \end{definition}        

        \begin{definition}[正常返]
            \MarkovChain 是一马氏链,给定任意的$i \in \mathbb{E}$,若$f_{ii} = 1$,且 $ \mu_{ii} < \infty$,则称状态 $i$ 是正常返的。
        \end{definition}

        \begin{definition}[遍历]
            \MarkovChain 是一马氏链,给定任意的$i \in \mathbb{E}$,若 $f_{ii} = 1$,且 $ \mu_{ii} < \infty$, 且 $d_i = 1$, 则称状态 $i$ 是遍历的。
        \end{definition}

        \begin{definition}[零常返]
            \MarkovChain 是一马氏链,给定任意的$i \in \mathbb{E}$,若 $f_{ii} = 1$,且 $ \mu_{ii} = \infty$,则称状态 $i$ 是零常反的。
        \end{definition}

        \begin{definition}[非常返]
            \MarkovChain 是一马氏链,给定任意的$i \in \mathbb{E}$,若 $f_{ii} < 1$,则称状态 $i$ 是非常返的。
        \end{definition}

    \subsection{转移概率的极限状态与平稳分布}
        
        \begin{definition}[极限分布]
            \MarkovChain 是一马氏链,记:
            \[\pi_j^* \triangleq \lim\limits_{n \to \infty} s_j(n) = \lim\limits_{n \to \infty} P(\textbf{X}_n = j)\]
            若 $\forall j \in \mathbb{E}, \pi_j^*$ 存在,且 $\sum\limits_{j \in \mathbb{E}} \pi_j^* = 1$,则称 $\pi = (\pi_1^*, \pi_2^*, \ldots)$ 为 \MarkovChain 的极限分布。
        \end{definition}

        \begin{definition}[平稳分布]
            \MarkovChain 是一马氏链,若:
            \[
                \left\{
                    \begin{array}{l}
                        \bm{\pi} \cdot \bm{1} = 1 \\
                        \bm{\pi} \cdot P = \bm{\pi}
                    \end{array}
                \right.
            \]
            则称 $ \bm{\pi} \triangleq \{\pi_j, j \in \mathbb{E}\}$ 为 \MarkovChain 的平稳分布。其中 $\bm{1}$为全$1$列向量, $P$ 为概率转移矩阵。
        \end{definition}