\section{布朗运动}
    \subsection{布朗运动的定义及基本性质}
        \begin{definition}[布朗运动]
            如果随机过程 \ContinuousMarkovChain 满足:
            \begin{enumerate}[label=(\arabic*).]
                \item $\textbf{X}(0) = 0$
                \item \ContinuousMarkovChain 有平稳独立增量
                \item $\forall t > 0, \textbf{X}(t) \sim N(0, \sigma^2 t)$
            \end{enumerate}
            则称 \ContinuousMarkovChain 为布朗运动。
        \end{definition}
    \subsection{布朗运动的首中时和最大值}
        \begin{definition}[首中时]
            记布朗运动 \ContinuousMarkovChain 首次击中 $x$ 的时刻为:
            \[ \textbf{T}_x = \inf\{t \geq 0, \textbf{X}(t) = x\} \]
        \end{definition}

        \begin{definition}[最大值]
            记布朗运动 \ContinuousMarkovChain 在 $[0, t]$ 中达到的最大值为:
            \[ \textbf{M}(t) = \max_{0 \leq s \leq t} \textbf{X}(s) \]
        \end{definition}
    \subsection{布朗运动的推广}
        \begin{definition}[带有线性漂移的布朗运动]
            若随机过程 \ContinuousMarkovChain 满足:
            \begin{enumerate}[label=(\arabic*)]
                \item $ \textbf{X}(0) = 0 $
                \item \ContinuousMarkovChain 有平稳独立增量
                \item $ \forall t \geq 0, \textbf{X}(t) \sim N(\mu t, \sigma^2 t) $
            \end{enumerate}
        \end{definition}

        \begin{definition}[几何布朗运动]
            设 $\{\textbf{W}(t), t \geq 0\}$是均值为 $\mu t$,方差参数为 $\sigma^2$ 的布朗运动,若:
            \[\textbf{X}(t) = e^{\textbf{W}(t)}\]
            则称 \ContinuousMarkovChain 为几何布朗运动。
            
        \end{definition}